\chapter*{Введение}
\addcontentsline{toc}{chapter}{Введение}
По мере развития научных познаний и вычислительной техники человечество сталкивается с новыми запросами, которые необходимы для решения различных задач. Для автоматизации процессов человек создал вычислительную машину — компьютер и нашел ему множество применений в различных областях.

В последние десятилетия машинная графика приобрела высокую популярность. Появляется запрос на мониторинг и управление окружающей средой. Требовались различные средства для удовлетворения таких потребностей. Со временем в машинной графике стало выделяться такое направление как 3d-моделирование и визуализация ландшафта.

Ландшафт представляет собой видимые особенности участка земли и его формы рельефа. Создание его вручную занимает определенное число ресурсов, на это тратится время. В связи с этим возникает потребность в программном обеспечении, которое позволило бы автоматизировать эти процессы, чтобы можно было быстро работать с такой моделью.

Целью данного курсового проекта является разработка программного обеспечения для визуализации трехмерного ландшафта. Для достижения поставленной цели необходимо решить следующий набор задач:
\begin{itemize}
	\item выполнить формализацию объектов синтезируемой сцены;
	\item провести исследование существующих алгоритмов решения поставленной цели;
	\item выбрать и описать подходящие алгоритмы для визуализации сцены и поставленной задачи;
	\item привести схемы используемых алгоритмов;
	\item описать использующиеся структуры данных;
	\item определить средства реализации ПО;
	\item реализовать ПО;
	\item выполнить исследование временных характеристик алгоритмов визуализации сцены.
\end{itemize}