\chapter*{Определения}
\addcontentsline{toc}{chapter}{Определения}
В данной расчетно-пояснительной записке применяются следующие термины с соответствующими определениями:
\begin{itemize}
	\item октава - повторение;
	\item лакунарность - величина, контролирующая значение частоты;
	\item постоянство - величина, контролирующая изменение амплитуды;
	\item зерно - значение, которое позволяет сформировать новые значения таблицы перестановок в алгоритме шума Перлина.
\end{itemize}